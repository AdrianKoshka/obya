%%%%%%%%%%%%%%%%%%%%%%%%%%%%%%%%%%%%%%%%%
% Beamer Presentation
% LaTeX Template
% Version 1.0 (10/11/12)
%
% This template has been downloaded from:
% http://www.LaTeXTemplates.com
%
% License:
% CC BY-NC-SA 3.0 (http://creativecommons.org/licenses/by-nc-sa/3.0/)
%
%%%%%%%%%%%%%%%%%%%%%%%%%%%%%%%%%%%%%%%%%

%----------------------------------------------------------------------------------------
%   PACKAGES AND THEMES
%----------------------------------------------------------------------------------------

\documentclass[handout]{beamer}

\mode<presentation> {
\usetheme{default}
}

\usepackage{graphicx} % Allows including images
\usepackage{booktabs} % Allows the use of \toprule, \midrule and \bottomrule in tables
\usepackage{hyperref} % Allows for the use for URLs
\usepackage{fontspec}
\usepackage{float} % Allow for the use of floats

% load external font
\usepackage[sfdefault]{FiraSans}
%----------------------------------------------------------------------------------------
%   TITLE PAGE
%----------------------------------------------------------------------------------------

\title[Our Bits, Your Applications!]{Flatpak: Our Bits, Your Applications!} % The short title appears at the bottom of every slide, the full title is only on the title page

\author{Adrian Lucr\`{e}ce C\'{e}leste} % Your name
\institute[Freelance] % Your institution as it will appear on the bottom of every slide, may be shorthand to save space
{
% \\ % Your institution for the title page
%\medskip
\emph{adrianlucrececeleste@airmail.cc} % Your email address
}
\date{\today} % Date, can be changed to a custom date

\begin{document}

\begin{frame}
\titlepage{} % Print the title page as the first slide
\end{frame}

\begin{frame}
\frametitle{Overview} % Table of contents slide, comment this block out to remove it
\tableofcontents % Throughout your presentation, if you choose to use \section{} and \subsection{} commands, these will automatically be printed on this slide as an overview of your presentation
\end{frame}

%----------------------------------------------------------------------------------------
%   PRESENTATION SLIDES
%----------------------------------------------------------------------------------------

%------------------------------------------------
\section{Introduction}
%------------------------------------------------
\subsection{Who am I?}
%------------------------------------------------
\begin{frame}
	\frametitle{Who am I?}
	\begin{itemize}
		\item{Linux user for 5--6 years}
		\item{Have an interest in security}
		\item{I maintain the Thunderbird flatpak on Flathub}
		\item{Working on getting Barrier and remote-viewer / virt-viewer packaged}
		\item{Unlikely package maintainer}
	\end{itemize}
\end{frame}
\note{
	\begin{itemize}
		\item{Intro to security was hardening, 2fa, and sandboxing}
		\item{Didn't have much luck with deb or RPM}
		\item{Never saw myself getting into package maintainership}
	\end{itemize}
}
%------------------------------------------------
\subsection{What is flatpak?} % A subsection can be created just before a set of slides with a common theme to further break down your presentation into chunks

\begin{frame}
\frametitle{What is Flatpak?}
	\begin{large}
		Distro agnostic package format.
	\end{large}
\medskip
	\begin{itemize}
		\item{Control over dependencies}
		\item{Easy build tools}
		\item{No vendor lock-in}
		\item{Runtimes}
		\item{Security}
			\begin{itemize}
				\item{Containerization}
				\item{``Portals''}
				\item{Sandboxing}
			\end{itemize}
	\end{itemize}
\end{frame}
%------------------------------------------------
\section{Flatpak Concepts}
\subsection{Runtimes}
%------------------------------------------------
\begin{frame}
	\frametitle{``Our Bits'' (Runtimes)}
	\begin{quotation}
	\ldots{}Collections of dependencies, which can be used by applications, and which can take a lot of the work out of application development\ldots
	\end{quotation}
	\medskip
	\begin{Large}3 major runtimes (at the time of writing)\end{Large}
	\begin{itemize}
		\item{Freedesktop}
		\item{GNOME}
		\item{KDE}
	\end{itemize}
\end{frame}
%------------------------------------------------
\subsection{Portals}
%------------------------------------------------
\begin{frame}
	\frametitle{Portals}
	\begin{quotation}
		Portals are high-level session bus APIs that provide selective access to resources to sandboxed applications.
	\end{quotation}
	\medskip
	\begin{Large}Examples of Portals:\end{Large}
		\begin{itemize}
			\item{Filesystem access}
			\item{Network access}
			\item{X11/Wayland}
			\item{PulseAudio}
			\item{OpenGL}
			\item{IPC}
		\end{itemize}
\end{frame}
%------------------------------------------------
\subsection{No Vendor Lock-in}
%------------------------------------------------
\begin{frame}
	\frametitle{No vendor lock-in (You have all the tools)}
	\begin{itemize}
		\item{Host your own repositories}
		\item{Create your own runtime(s)}
	\end{itemize}
	\begin{figure}[h]\includegraphics[scale=0.25]{tools.jpg}\end{figure}
\end{frame}
\note{
	\begin{itemize}
		\item{Unlike snap, anyone can host a repo}
	\end{itemize}
}
%------------------------------------------------
\subsection{Security}
%------------------------------------------------
\begin{frame}
	\frametitle{Security}
	\begin{itemize}
		\item{Principle of least privilege}
		\item{\href{https://github.com/projectatomic/bubblewrap}{Bubblewrap}}
		\item{Portals}
		\item{\href{https://github.com/flatpak/flatpak/wiki/Sandbox}{Specifics about sandboxing model}}
	\end{itemize}
	\begin{figure}[h]\includegraphics[scale=0.25]{sandbox.jpg}\end{figure}
\end{frame}
\note{
	\begin{itemize}
		\item{All processes run as the user with no capabilities}
		\item{User namespaces}
		\item{Fileystem namespace}
		\item{Ran as transient systemd user scope}
		\item{Limited /dev, /etc, / is a private tmpfs, as is /dev/shm}
		\item{All mounts are no suid}
		\item{All mounts nodev sans /dev and /dev/pts}
		\item{/usr is a bind-mount of the runtime}
	\end{itemize}
}
%------------------------------------------------
\subsection{Easy Build Tools}
%------------------------------------------------
\begin{frame}
	\frametitle{Easy Build Tools}
	\begin{itemize}
		\item{Manifest file in JSON or YAML}
		\item{Appstream validation post-build}
		\item{flatpak-builder makes building very easy™}
			\begin{itemize}
				\item{Assuming upstream project use a sane build system}
			\end{itemize}
		\item{Even I can do it™}
	\end{itemize}
\end{frame}
\note{
	Manifest contains info about:
	\begin{itemize}
		\item{Sandbox permissions}
		\item{Runtime version}
		\item{Build options}
	\end{itemize}
	\medskip
	\begin{itemize}
		\item{flatpak-builder is just a wrapper}
		\item{Builds should just be: ./configure;make;make install, or meson, etc}
	\end{itemize}

}
%------------------------------------------------
\section{My Experience with Flatpak}
%------------------------------------------------
\begin{frame}
	\frametitle{Short overview of packaging Thunderbird}
	\begin{itemize}
		\item{Find out someone before me had done 99\% of the work}
		\item{Notice they hadn't got it up on flathub}
		\item{Fork their repo}
		\item{Polish things up a bit}
		\item{Read flathub submission guidelines}
		\item{Submit PR to flathub}
		\item{Work with the flathub team to really polish it up}
		\item{\ldots}
		\item{Finally get PR accepted and merged!}
	\end{itemize}
\end{frame}
%------------------------------------------------

%------------------------------------------------
\begin{frame}
	\frametitle{Short overview of my progress in packaging Barrier}
	\begin{itemize}
		\item{Create a rough-draft manifest}
		\item{99\% of dependencies are in the KDE runtime}
		\item{Learn avahi libraries needed aren't in the KDE runtime.}
		\item{Try to learn how to build avahi, contact an avahi maintainer, eventually get the libraries built}
		\item{Learn not all projects using auto-tools use it, correctly}
		\item{Get Barrier to at least build + run properly}
		\item{Write a rough-draft appstream file}
		\item{Pester upstream to be more XDG compliant}
	\end{itemize}
\end{frame}
%------------------------------------------------

%------------------------------------------------
\begin{frame}
	\Large{\centerline{I hope this inspires you to give flatpak a try!}}
\end{frame}

%------------------------------------------------

%------------------------------------------------
\begin{frame}
\Huge{\centerline{The End}}
\end{frame}

%-------------------------------------------------------------------------------

\end{document}
